\documentclass[12pt]{article}

\usepackage[utf8]{inputenc}
\usepackage[margin=3cm]{geometry}
\usepackage{amsmath, amssymb}
\usepackage{listings}
\usepackage{paralist}

\newcommand\abs[1]{\left|#1\right|}
\newcommand{\R}{\mathbb{R}}
\newcommand\ttt\texttt

\setlength{\parindent}{0pt}

\begin{document}

\section*{Brachistochrone}

\subsection*{Introduction}

The \emph{Brachistochrone} problem asks for the curve that a particle needs to
follow, only under the presence of gravity, to go from point $A$ to $B$ in the
least amount of time.

We will solve this problem with the help of SCIP.

\subsection*{Problem formulation}

Assume $A = (0,1)$, $B = (1,0)$. Given a curve $y(x)$ such that $y(0) = 1, y(1) = 0$, we can assign to it
the time that a particle dropped at $A$ takes to get to $B$ following $y(x)$. Let us call this time $T$
(though $T_y$ would be better).

To find out $T$, notice that time is distance covered divided by the velocity.
Let us find the distance covered first. For that write the curve as $x \to (x, y(x))$.
The length covered from $x = 0$ to $x = \bar x$ is $s(\bar x) = \int_0^{\bar x} \sqrt{1 + {y^{'}}^2(\alpha)} d \alpha$.

Now, $v(t) = \tfrac{ds}{dt}$ and so
\[
    T = \int_0^T dt
\]
Using the change of variable (abussing notation)
$t = t(s)$ and noting that $t = 0 \Leftrightarrow s = 0$ and $t = T \Leftrightarrow s = s(1)$, we get
\[
    T = \int_0^T dt = \int_0^{s(1)} \frac{dt}{ds} ds = \int_0^{s(1)} \frac{1}{v(s)} ds
\]
Finally, using the change of variables $s = s(x)$, we obtain
\[
    T = \int_0^T dt = \int_0^{s(1)} \frac{1}{v(s)} ds = \int_0^{1} \frac{\sqrt{1 + {y^{'}}^2(x)}}{v(x)} dx
\]

To find $v$ we can use that the energy of the system is constant.
The energy is the sum of the kinetic ($\tfrac{1}{2} m v^2$) and potential energy ($m g y$).
At the beginning, the energy is $E(0) = \tfrac{1}{2} m v(0)^2 + m g y(0) = m g$.
Then at $x$, $E(x) = \tfrac{1}{2} m v(x)^2 + m g y(x)$ and since it is constant, $E(0) = E(x)$, which implies
$v = \sqrt{2g(1 - y)}$

Then, given a curve $y$, we obtain
\[
    T = \int_0^1 \frac{\sqrt{1 + {y^{'}}^2}}{\sqrt{2g(1 - y)}} dx
\]
and our task is to find the curve $y$ that minimizes $T$,
\[
    \min_{y} \int_0^1 \frac{\sqrt{1 + {y^{'}}^2}}{\sqrt{2g(1 - y)}} dx
\]

With techniques from calculus of variations this problem can be solved analytically.
The optimal curve, for this particular case, is given parametrically on $\theta$ by
% other possible values are
% A = 0,0
% B = 3,-2
% constant = 1.00133
% theta up to 3.069
% time = 0.98
\begin{align*}
    x(\theta) &= 0.573 (\theta - \sin(\theta)) \\
    y(\theta) &= -0.573 (1 - \cos(\theta)) \\
       \theta &\in [0, 2.412]
\end{align*}
and the total time is $T = 0.583$

\subsection*{Discretization}

To use SCIP we need to discretize the problem.
For this, we divide the integral in small pieces:

\[
    \int_0^1 \frac{\sqrt{1 + {y^{'}}^2}}{\sqrt{2g(1 - y)}} dx = \sum_{i = 0}^{N-1} \int_{x_i}^{x_{i+1}} \frac{\sqrt{1 + {y^{'}}^2}}{\sqrt{2g(1 - y)}} dx
\]

Approximating $y(x)$ by a linear for $x \in [x_i, x_{i+1}]$, we can write it as $y(x) = m_i (x - x_i) + y_i$,
where $m_i = \tfrac{y_{i+1} - y_i}{x_{i+1} - x_i}$.
With this, one can show that
%\[
%    \int_{x_i}^{x_{i+1}} \frac{\sqrt{1 + {y^{'}}^2}}{\sqrt{2g(1 - y)}} dx
%    = \frac{\sqrt{1 + m_i^2}}{\sqrt{2g}} \int_{x_i}^{x_{i+1}} \frac{1}{\sqrt{1 - y}} dx
%\]
%
%Let us compute the integral
%\begin{align*}
%    \int_{x_i}^{x_{i+1}} \frac{1}{\sqrt{1 - m_i(x-x_i) - y_i}} dx
%    &= -\frac{2}{m_i} \sqrt{1 - m_i(x-x_i) - y_i}\Big|_{x_i}^{x_{i+1}} \\
%    &= -\frac{2}{m_i} (\sqrt{1 - y_{i+1}} - \sqrt{1 - y_{i}}) \\
%\end{align*}
%
%So putting all together and rationalizing the expression $(\sqrt{1 - y_{i+1}} - \sqrt{1 - y_{i}})$ we obtain
\[
    \int_0^1 \frac{\sqrt{1 + {y^{'}}^2}}{\sqrt{2g(1 - y)}} dx
    \approx
    \sqrt{\frac{2}{g}} \sum_{i = 0}^{N-1} \frac{\sqrt{(x_{i+1} - x_i)^2 + (y_{i+1} - y_i)^2}}{\sqrt{1 - y_{i+1}} + \sqrt{1 - y_{i}}}
\]

\subsection*{Using SCIP}

Your task is to write a model to solve the discretized problem with PySCIPOpt.

(Optional) Show how to obtain the discretization.

\end{document}

