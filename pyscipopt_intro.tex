\documentclass[11pt]{article}

\usepackage[utf8]{inputenc}
\usepackage[margin=3cm]{geometry}
\usepackage{amsmath, amssymb}
\usepackage{listings}

\newcommand\abs[1]{\left|#1\right|}
\newcommand{\R}{\mathbb{R}}
\newcommand\ttt\texttt
\renewcommand\arraystretch{1.5}

\setlength{\parindent}{0pt}

\title{Introduction to PySCIPOpt}

\begin{document}

\maketitle

PySCIPOpt is the Python interface for SCIP.
It allows for \emph{fast model prototyping} with \emph{flexible expressions}.
Additionally PySCIPOpt supports \emph{user-plugins} similar to C.

\lstset{language=python,%
basicstyle=\sffamily\footnotesize,%
numberstyle=\sffamily\tiny\color{siennabrown},stepnumber=1}
\begin{lstlisting}[frame=tb]{}
  from pyscipopt import Model

  # initialize model
  m = Model("svm")

  # add variables
  x = m.addVar(vtype='B', name='x')
  y = m.addVar(vtype='C', name='y')

  # add a constraint
  m.addCons(3 * x + 2 * y >= 4)

  m.optimize()
\end{lstlisting}

\section*{Adding variables}

Different variable types \ttt{vtype} are supported by \ttt{addVars}: \ttt{B} (binary), \ttt{I} (integer), \ttt{C} (continuous).
You can specify lower and upper bounds: \ttt{lb} (default: 0), \ttt{ub} (default: \ttt{None} $\sim \infty$), as well as objective coefficients: \ttt{obj} (default: $0.0$).

\lstset{language=python,%
basicstyle=\sffamily\footnotesize,%
numberstyle=\sffamily\tiny\color{siennabrown},stepnumber=1}

\begin{lstlisting}[frame=tb]{}
  m = Model("svm")

  # add variables
  x = m.addVar(vtype='C', name='x', lb=-10, ub=10, obj = 1.0)
\end{lstlisting}

\section*{Nonlinear objective functions}

SCIP can only handle linear objective functions.
Therefore a nonlinear objective function must be transformed into a constraint using an auxiliary variable:
$$ \text{min } f(x) \; \Leftrightarrow \; \text{min } t \text{ such that } f(x) \leq t $$

\section*{Adding constraints}

\ttt{addCons} understands expression objects and requires an operator \ttt{<=}, \ttt{>=}, \ttt{==}. Variables and constants can appear on both sides of the operator.
For complex expressions use the \ttt{quicksum} summation and python list comprehension \ttt{[a[i] for i in range(5)]}, multiplying variables yields nonlinear constraints,

\begin{lstlisting}[frame=tb]{}
  from pyscipopt import Model, quicksum
  m = Model("svm")
  vars = []
  for i in range(10):
      vars.append(m.addVar(vtype='B', name=str(i)))
  m.addCons(vars[0] ** 2 == 1)
  m.addCons((quicksum(i * vars[i] for i in range(10)) <= 15),
            name="mycons")

  m.optimize()
\end{lstlisting}

\end{document}

